\documentclass{article}
\usepackage[T1]{fontenc}
\usepackage{charter}
%\usepackage{helvet}
%\renewcommand{\familydefault}{\sfdefault}
\usepackage{microtype}
\pagestyle{empty}
\linespread{1.2}
\begin{document}
\frenchspacing
\noindent
{\LARGE 7 \TeX{} Gyre Collection}\\
~\\
The \TeX{} Gyre Collection is the open source counterpart of the proprietary
Post\-Script ``core font set'' (Avant Garde, Bookman, Courier, Helvetica, New
Century Schoolbook, Palatino, Times Roman, and Zapf Chancery). It was
created from the clones of this set that URW donated to the the free
software community (Gothic, Bookman, Nimbus Mono, Nimbus Sans, Century
Schoolbook, \mbox{Palladio}, Nimbus Roman, Chancery). In the same order,
these are the new names given to them by the \TeX{} Gyre project: Adventor,
Bonum, Cursor, Heros, Schola, Pagella, Termes, and Chorus.  Get it?

Orginally, \LaTeX{} support for these fonts was provided by the PSNFSS
(Post\-Script New Font Selection Scheme) project that conveniently ignored
the fact that the fonts it uses aren't really Adobe's versions but rather
URW's versions.  That's why you can still simply \textbackslash
usepackage\{avant, bookman, courier, helvet, newcent, palatino, times,
and chancery\} to activate the (clones of) common PostScript fonts.
%For more information, see psnfss2e.pdf included in every modern \LaTeX
%distribution.

The \TeX{} Gyre project makes an end to this naivety with its new naming
scheme and makes a renewed effort to adapt and extend the core font set for
use with \LaTeX{}.  Only four of these font families are suited for running
text (instead of headings, listings, citations, etc.).  Here they are,
including the proper way of using them through the \TeX{} Gyre project:

\end{document}
