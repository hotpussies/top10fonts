\documentclass{article}
\usepackage{microtype}
\usepackage[T1]{fontenc}
\usepackage{gfsdidot}
\usepackage{inconsolata}
\pagestyle{empty}
\linespread{1.2}
\begin{document}
\frenchspacing

\noindent
{\LARGE 3 GFS Didot}\\
~\\
Didot is the ``father of modern fonts''. It was created by the famous French
printing and type producing family between 1784 and 1811.  Firmin Didot cut
the letters, and cast them as type in Paris.  His brother, Pierre Didot used
the types in printing.  Together, they are regarded the inventors of the
neoclassical \emph{Didone} style of typefaces, evocative of the Age of
Enlightenment.

The typeface is characterized by serifs without brackets, vertical
orientation of weight axes, strong contrast between thick and thin lines,
and an unornamented, ``modern'' appearance, inspired by John
Baskerville's earlier experiments with increasing stroke contrast and a more
condensed armature.
%Didot resembles the typefaces developed by Giambattista Bodoni in Italy.

The \TeX{} version of Didot was created by the \emph{Greek Font Society}, a
non-profit organization in Greece, founded in 1992, devoted to improving the
standard of Greek digital typography.  Other famous typefaces recreated by
the GFS include Artemisia, Neohellenic and Bodoni.  To use them, put one of
the following directives in your document header.

\textbackslash usepackage\{gfsartemisia\}

\textbackslash usepackage\{gfsbodoni\}

\textbackslash usepackage\{gfsdidot\}

\textbackslash usepackage\{gfsneohellenic\}

\end{document}
