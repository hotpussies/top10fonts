\documentclass{article}
\usepackage{microtype}
\usepackage[T1]{fontenc}
\usepackage{fourier}
\pagestyle{empty}
\linespread{1.2}
\begin{document}
\frenchspacing

\noindent
{\LARGE 4 Utopia}\\
~\\
Utopia is a transitional serif typeface designed by Robert Slimbach, and
first released in 1989 by Adobe.  Its looks are consistent, formal, and very
clearly readable even on low-resolution media such as computer printers and
screens.

Adobe's release of Utopia was a response to Bitstream's release of the
Charter typeface (\textbackslash usepackage\{charter\}) in 1987,
another highly readable set of fonts designed specifically for
low-resolution printers.  Both corporations donated their fonts to the
X Consortium in 1992 in order to gain popularity.  Adobe, however, did
so using very restrictive license, causing concerns in the free
software community.  (This is the reason you see a ``This package is
to be regarded as obsolete'' warning when you try to \textbackslash
usepackage\{utopia\} in \LaTeX.)

In 2006, Adobe re-released Utopia, this time under a truly free license, and
since then it has been available through at least two packages: the
Fourier-GUTenberg project (\textbackslash usepackage\{fourier\}), and
\emph{mathdesign} (\textbackslash usepackage[utopia]\{mathdesign\}).

\end{document}
