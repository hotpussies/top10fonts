\documentclass{article}
\usepackage{microtype}
\usepackage[OT1]{fontenc}
\usepackage{librebaskerville}
\usepackage{boisik}
\usepackage{mflogo}
\usepackage[utf8]{inputenc}
\pagestyle{empty}
\linespread{1.2}
\begin{document}
\frenchspacing
\noindent
{\LARGE $10$ Boisik}~\\
~\\
Boisik is without a doubt the ugliest font available in the \LaTeX{}
distribution.  One look at this paragraph and you'll agree that the
last place in this Top 10 is a well-deserved one.

However, Boisik is very much worth mentioning because it has
completely been written in the archaic \MF{} system.  That's right,
every parameter of this font is globally adjustable, and every font
size has different glyphs.  Boisik is inspired by the classic 18th
century {\librebaskerville\small Baskerville}, which is itself
available with \textbackslash{}usepackage\{librebaskerville\}. Boisik
can be selected by \textbackslash{}usepackage\{boisik\}.

Boisik
is a tribute to the original Computer Modern typeface, and it's
ugliness only underwrites the geniousness of Knuth's creation.


\end{document}
