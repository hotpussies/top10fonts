\documentclass{article}
\usepackage{microtype}
\usepackage[T1]{fontenc}
\usepackage[utf8]{inputenc}
\usepackage{charter}
\pagestyle{empty}
\setlength{\parindent}{0pt}
\setlength{\parskip}{1em}
\linespread{1.2}
\begin{document}
\frenchspacing

{\LARGE 9 Bitstream Charter}

Bitstream Charter is a glyphic serif typeface designed by Matthew Carter in
1987 for Bitstream Inc., a digital type foundry.  Bitstream Charter is a
typeface optimized for printing on the low-resolution 300 dpi laser printers
of the 1980s.  The typeface is suitable for printing on both modern
high-resolution laser printers and lower resolution inexpensive inkjet
printers.

In 1992, along with their version of Courier, Bitstream donated the Charter
font to the X Consortium under terms that allowed modified versions of the
font to be redistributed.

\end{document}
