\documentclass{article}
\usepackage[T1]{fontenc}
\usepackage{lmodern}
\usepackage{inconsolata}
\usepackage{microtype}
\usepackage{mflogo}
\pagestyle{empty}
\linespread{1.2}
\begin{document}
\frenchspacing

\noindent
{\LARGE 1 Computer Modern}\\
~\\
% Some love it, some hate it, but fact is that a
\noindent
The users of any \TeX{} system get the Computer Modern typeface by default in
all their documents.  This idiosyncratic font makes \TeX{} documents
instantly recognizable to anyone who's ever used this system and instantly
impresses anyone else with its timeless design.  It has very high contrast
between thick and thin elements, very consistent and characteristic
strokes, and relatively short ascenders and descenders.  Of course, it's
classified as ``modern''.

Computer Modern was created by Donald Knuth, the great mathematician and
creator of \TeX, and initially released in $1978$ together with \TeX{} and
\MF.  The latter is the programming language that he used to create Computer
Modern.  In this language, the Computer Modern shapes are described by 62
distinct parameters that allow the glyphs to be changed generically.  The
idea is that certain aspects of the font can be adjusted by typesetters to
match the type of material.  In reality, no one besides Knuth has been able
to create such a mature and widely-used type family as Computer Modern using
the \MF{} system, although you'll see one more \MF{} typeface further down
this page.

To use this font in \LaTeX, you don't have to do anything. However, you might consider
to \textbackslash usepackage\{lmodern\} to obtain Latin Modern,
a PostScript version of Computer Modern created with \MP.  It looks
(almost) exactly the same, but provides more convenient scaling of glyphs
among other small improvements.  But rest assured that \LaTeX{}, by default,
gives you the best typeface available.

\end{document}
